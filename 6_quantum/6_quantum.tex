\renewcommand{\lastmod}{November 14, 2023}
\renewcommand{\chapterauthors}{Markus Lippitz}

\chapter{Hybridization of quantum mechanical systems}
\label{chap:hybrid_quantum}

\section*{Overview}

In this chapter we discuss the coupling of systems to be described by quantum mechanics. We start with a toy model to get used to the formalism, and then come to the Hückl method, which is close to the chemical hybridization of atomic orbitals. Here the coupling comes from the overlap of the atomic wave functions. As a second example, we will consider coupling due to the interaction of optical transition dipole moments, leading to so-called molecular H- and J-aggregates. This is the quantum mechanical analog of the interaction of scattering particles presented in the previous chapter.

%   * exciton polariton ?
%  * temporal evolution of superposition states ?




\section{Variational principle}

The Schrödinger equation
\begin{equation}
 \hat{H} \ket{\Phi} = E_0 \, \ket{\Phi} 
\end{equation}
is a differential equation and not always easy to solve.  This is where the variational principle comes in.
It says that for an arbitrary wave function $\ket{\Psi}$ we always have
\begin{equation}
 E = \frac{\braket{\Psi | H | \Psi}} {\braket{\Psi | \Psi}} \ge E_0 \quad .
 \label{eq:6_variation}
\end{equation}
$E$ becomes minimal if $\ket{\Psi}$ solves Schrödinger's equation. But even if $\ket{\Psi}$ is not a solution of the Schrödinger equation, one can easily calculate Eq.~\ref{eq:6_variation}. So we try different test functions and try to minimise the energy according to Eq.~\ref{eq:6_variation}. This way we get closer and closer to the true eigenfunction, which is the solution of the Schrödinger equation. Unfortunately, we do not know if we could get even smaller values of $E$ by using even better test functions.


We want to investigate what happens when we couple quantum mechanical systems. We already know the solutions $\phi_i$ of the individual system, so we try to express the new coupled system by a linear combination of the known individual parts:
\begin{equation}
 \ket{\Psi} = \sum_i  c_i \ket{\phi_i}  \label{eq:6_psi}
\end{equation}
with normalized $\ket{\phi_i}$ and real-valued coefficients $c_i$. This gives
\begin{eqnarray}
\braket{\Psi | \Psi} &= & \sum_i c_i^2 + \sum_{i,j} c_i c_j \underbrace{\braket{\phi_i| \phi_j}}_{= S_{ij}}\\
\braket{\Psi | H | \Psi} &=& \sum_i c_i^2 \underbrace{\braket{\phi_i | H | \phi_i }}_{= H_{ii}} 
                     + \sum_{i,j} c_i c_j \underbrace{\braket{\phi_i | H | \phi_j }}_{= H_{ij}}  \quad .
\end{eqnarray}
$S_{ij}$ are the respective overlap integrals of the two wavefunctions and $H_{ij}$ the matrix elements of the Hamilton operator.  The diagonal elements $H_{ii}$  give the Coulomb energy, and the off-diagonal elements $H_{ij}$  the exchange energy. With these abbreviations, the self-energy can be written as
\begin{equation}
  E = \frac{\sum_i c_i^2 H_{ii}  + \sum_{i,j} c_i c_j H_{ij}}{1 + \sum_{i,j} c_i c_j S_{ij}}
  \quad . \label{eq:6_e_variation}
\end{equation}
For a minimum self-energy $E$, the partial derivatives to $c_i$ must both be zero. This can be written as 
\begin{equation}
   \left|   \mathbf{H} - E \mathbf{S} \right| = 0  \quad .
\end{equation}
The eigen-energies $E$ are solutions to this equation.



% After a few transformations, we find two solutions $E_\pm$ for the minimum energy $E$ as
% \begin{equation}
%  \begin{vmatrix}
%    H_{11} - E & H_{12} - E \, S \\ H_{12} - E \, S & H_{22} - E \\
%  \end{vmatrix}
% = 0
% \quad
% \text{or} \quad
% E_\pm = \frac{H_{11} \pm H_{12}}{1 \pm S} \quad ,
% \end{equation}
% where in the last step we assumed that $H_{11} = H_{22}$. In this case, the coefficients $c_i$ are.
% \begin{equation}
% c_1 = \pm c_2 = \frac{1}{\sqrt{2 (1 \pm S)}} \quad ,
% \end{equation}
% because yes ${\braket{\Psi | \Psi}} = c_1^2 + c_2^2 + 2 c_1 c_2 S = 1$ should be.



\section{Two coupled states}

Let us start with only two states $\psi_1$ and $\psi_2$. For simplicity, we label the diagonal entries of $\mathbf{H}$ as $E_i$ and the off-diagonal entries as $H_{ij} = J$, i.e. we assume that both are identical. Without coupling ($J=0$), the Hamiltonian reads as a matrix
\begin{equation}
\hat{H}_0 = \begin{pmatrix} E_1 & 0 \\ 0 & E_2 \end{pmatrix} 
     \quad .
\end{equation}
 When the two states are coupled, then the energy of one state somehow depends on the other. In the matrix this results in an  off-diagonal element $J$
\begin{equation}
\hat{H}_{coupled} = \begin{pmatrix} E_1 & J \\ J & E_2 \end{pmatrix} 
\quad . 
\end{equation}
As a consequence, the original eigen-functions $\psi_i$ are no longer eigen-functions of this coupled Hamilton operator. We find new eigen-functions and eigen-values by diagonalizing $\hat{H}_{coupled}$, so that the diagonal elements become
\begin{equation}
 E_\pm = \frac{E_1 + E_2}{2} \pm \sqrt{ \left( \frac{E_1 - E_2}{2} \right)^2 + J^2 }
\end{equation}
and the new eigen-functions are\footcite[eq. 8.10]{Parson}
\begin{equation}
 \psi_{\pm} = 
\sqrt{\frac{1 \pm s}{2}} \, \psi_1 \, \, \pm \, \, \sqrt{\frac{1 \mp s}{2}}  \, \psi_2 \quad ,
\end{equation}
with
\begin{equation}
s = \frac{E_1 - E_2}{\sqrt{(E_1 - E_2)^2 + (2J)^2}} \quad .
\end{equation}
We can distinguish two limiting cases. The coupling energy $J$ can be larger than the energy difference between the two states, i.e. $|J| \gg |E_1 - E_2| / 2$. Then then new eigen-energies are split up by $\pm J$ around the average of the old eigen-energies $(E_1 + E_2) /2$. The eigen-functions in this situation are symmetric and anti-symmetric combinations of the old eigen-function, i.e. $\psi_\pm = \pm \psi_1 + \psi_2$. When the coupling energy is small, i.e. $|J| \ll |E_1 - E_2| / 2$, then the new eigen-energies and eigen-functions are close to the old ones.

\begin{figure}
   \inputtikz{\currfiledir anticrossing_v2}
\caption{Eigen-energies and weights of the eigen-functions as function of the unperturbed energies ($E_2 = 1$).}
\label{fig:6_anticrossing}
\end{figure}








\section{The Hückel method}

A classic example of hybridized quantum mechanical wave functions is the Hückel method for describing aromatic hydrocarbons.
 In conjugated molecules, the mechanical framework is formed by $\sigma$ bonds between carbon atoms. A chain of carbon atoms is further linked by alternating $\sigma$ and $\pi$ bonds. The electrons involved in these bonds are then delocalised throughout the chain. The Hückel approximation can be used to calculate these extended $\pi$ orbitals.

Thus, we consider only a subset of all atomic orbitals, only the $\pi$ orbitals that also participate in the $\pi$ bond. We assume that
\begin{itemize} \setlength{\itemsep}{0pt}
\item the atomic orbitals overlap only with themselves, so $S_{ij} = \delta_{ij}$
\item all atoms are identical, so $H_{ii} = \alpha$
\item exchange takes place only between adjacent orbitals, so $H_{ij} = \beta < 0 $ if atoms $i$ and $j$ are adjacent, otherwise $0$. 
\end{itemize}

Analogous to equation \ref{eq:6_e_variation} above, we calculate the self-energy according to the variation principle
\begin{equation}
 E = \frac{ \sum_{i,j} c_i \, c_j \, H_{i,j} }{ \sum_{i,j} c_i \, c_j \, S_{i,j} } \quad .
\end{equation}
The minimum self-energy $E$ is obtained when all partial derivatives to the $c_i$ are zero, or when
\begin{equation}
 \left| \mathbf{H} - E \, \mathbf{S}\right| = 0 \quad .
\end{equation}
Since we have assumed $S_{ij} = \delta_{ij}$, this simplifies to 
\begin{equation}
 \left| \mathbf{H} - E \, \mathds{1} \right| = 0 \quad .
\end{equation}
So we have to determine the eigenvalues and eigenvectors of $H_{i,j}$. The eigenvalues indicate the energy of the state, the eigenvectors the corresponding linear combination of the atomic orbitals.

As an example we consider benzene (\ch{C6H6}). The 6 carbon atoms are sp$^2$ hybridized. $\sigma$ bonds connect the carbon atoms with each other and with the hydrogen atoms. One non-hybridized p-orbital is perpendicular to each ring. These orbitals are considered in the Hückel approximation. The Hamiltonian matrix $H_{ij}$ then has the form (zeros omitted)
\begin{equation}
\mathbf{H} = 
 \begin{pmatrix}
  \alpha & \beta & & &  & \beta \\
  \beta & \alpha & \beta & & \\
  & \beta & \alpha & \beta & & \\
 & & \beta & \alpha & \beta & \\
& & & \beta & \alpha & \beta \\
\beta &  & & & \beta & \alpha 
 \end{pmatrix}  \quad .
\end{equation}
The $\beta$ in the corners close the ring.
If we assume $E = \alpha + x \beta$, then the eigenvalue equation simplifies to 
\begin{equation}
x^6 - 6 x^4 + 9x^2 - 4 = 0 \quad \text{or} \quad x = \pm 1, \pm 1, \pm 2 \quad .
\end{equation}
How to do this numerically you can see in the  
Pluto script\pluto{hueckel}.




\begin{marginfigure}[20mm]
\inputtikz{\currfiledir benzol}
\caption{Molecular orbitals of benzene in the Hückel approximation. The colors encode the sign of the wave function. The arrangement corresponds to the self-energy.\label{fig:6_benzene}}
\end{marginfigure}

Since we have to fill a total of 6 electrons into these orbitals, and each orbital can be occupied by 2 electrons (spin up and down), the orbitals with $E=\alpha + 2 \beta$ and the two orbitals with $E = \alpha + \beta$ are occupied\sidenote{$\beta < 0$}. Thus, these orbitals also contribute to the binding, since they reduce the total energy by $8\beta$ overall. Considering the eigenfunctions, we see that the orbital with $E=\alpha \pm 2 \beta$ is delocalized over the whole ring, the two with $E = \alpha \pm \beta$ over two atoms.

The Hückel approximation in molecular physics corresponds to the \emph{tight binding} method for calculating the band structure of electrons in solid state physics. In solid state physics, one makes the transition from here $N=6$ atoms to $N= 6 \cdot 10^{23}$ atoms, which then gives rise to $6 \cdot 10^{23}$ closely spaced states for electrons, all described by wave functions similar to Figure \ref{fig:6_benzene}.

%https://en.wikipedia.org/wiki/H%C3%BCckel_method#Delocalization_energy,_%CF%80-bond_orders,_and_%CF%80-electron_populations

\begin{questions} 
\item Compare the electron eigenfunctions of benzene in the Hückel approximation with those of a (possibly annular) box.
\end{questions}

\section*{Tight binding model}

Let us thus repeat the Hückel method, but with the solid state in mind, i.e. many atoms and using plane waves to describe everything. In the following, the tilde denotes variables that apply to a single atom. Let $\tilde{V}$ be the Coulomb-like potential of an atom. We know the solutions $\tilde{\psi}$ of the Schrödinger equation
\begin{equation}
    H_A \tilde{\psi} = \left( - \frac{\hbar^2}{2m} \nabla^2 + \tilde{V} \right) \tilde{\psi} = \tilde{E} \tilde{\psi} \quad .
\end{equation}
In the crystal there are now atomic nuclei at the lattice points $\mathbf{R}_m$, which cause an additional perturbation term in the Hamilton operator
\begin{equation}
    H_S = \sum_{m \neq n} \tilde{V}(\mathbf{r} - \mathbf{R}_m) \quad ,
\end{equation}
where the atom at $\mathbf{R}_n$ is already considered in the unperturbed operator. As an ansatz for the wave function, we choose a superposition of atom eigenfunctions at the locations $\mathbf{R}_m$ as in the Hückel method, i.e.
\begin{equation}
    \psi = \sum_m a_m \tilde{\psi} (\mathbf{r} - \mathbf{R}_m) \quad .
\end{equation}

Now we use the Bloch theorem\sidenote{It would also work without, but makes it easier here}: the lattice-periodic part $u_\mathbf{k}$ is provided by the wave functions $\tilde{\psi} (\mathbf{r} - \mathbf{R}_m)$, so the plane wave must be in the coefficients $a_m$, i.e.
\begin{equation}
    a_m \propto e^{i \mathbf{k} \cdot \mathbf{R}_m}
\end{equation}
with suitable normalization. This means that the wave function is 
\begin{equation}
    \psi = \frac{1}{\sqrt{N}} \sum_m \tilde{\psi} (\mathbf{r} - \mathbf{R}_m) \, e^{i \mathbf{k} \cdot \mathbf{R}_m} \quad .
\end{equation}
The self-energy is as always
\begin{equation}
    E = \frac{\int \psi^\star \, H \, \psi \, dV }{\int \psi^\star \, \psi \, dV} \quad .
\end{equation}
The denominator is close to one because the atomic wave functions overlap very little. The numerator is more interesting. We can draw the sums over the atomic positions in front of the integral 
\begin{equation}
    E = \frac{1}{N} \sum_{m,n} \, e^{i \mathbf{k} \cdot (\mathbf{R}_m - \mathbf{R}_n) }\,
    \int \tilde{\psi}^\star (\mathbf{r} - \mathbf{R}_n) \left[ H_A + H_S(\mathbf{r} - \mathbf{R}_m) \right] \tilde{\psi} (\mathbf{r} - \mathbf{R}_m) \quad .
\end{equation}
We can distinguish three contributions
\begin{itemize}
\item Integrands of the form $\tilde{\psi}^\star_n \, H_A \, \tilde{\psi}_n$. These are the eigen-energies of the atoms that we already know.
\item Integrands of the form $\tilde{\psi}^\star_n \, H_S \, \tilde{\psi}_n$. This is the influence of the potentials of the other atoms (in $H_S$) on 'our' atom $n$. We abbreviate this Coulomb integral with $-\alpha$.
\item integrands of the form $\tilde{\psi}^\star_n \, H_S \, \tilde{\psi}_m$. This is the influence of the overlap with the other wave functions. 
 We abbreviate this transfer integral with $-\beta_m$.
\end{itemize}    
In total, we thus have\sidenote{The sums over $\tilde{E}$ and $\alpha$ provide an $N$, which is canceled by the normalization}.
\begin{equation}
    E = \tilde{E} - \alpha - \sum_m \beta_m \, e^{i \mathbf{k} \cdot (\mathbf{R}_m - \mathbf{R}_n) }\, \quad .
\end{equation}
The Coulomb integral $\alpha$ causes a decrease in energy because the neighboring atoms also contribute some attractive Coulomb potential. The transfer integral $\beta$ can be both positive and negative, and also direction-dependent, as in the case of covalent bonding in molecular physics. There, too, we saw that s and p orbitals can provide attractive or repulsive energy contributions depending on their arrangement. Exactly the same thing happens here. This integral provides the dependence on the wave vector $\mathbf{k}$ and thus the dispersion relation.


\section{Example: cubic-primitive lattice}

As an example, we consider a cubic-primitive lattice, only allow interaction between nearest neighbors, and assume the interaction to be direction-independent, as it would be for atomic s-orbitals. For the term $\mathbf{R}_m - \mathbf{R}_n$, only the three (Cartesian) lattice vectors of length $a$, each with both signs, can be considered. Multiplying out the scalar product results in
\begin{equation}
    E = \tilde{E} - \alpha - 2 \beta \left[ \cos( k_x a ) + \cos( k_y a ) + \cos( k_z a ) \right] \quad .
\end{equation}
In total, this covers a band of width $\tilde{E} - \alpha \pm 6 \beta $. The band is cosine-shaped. At the $\Gamma$ point and at the boundary of the Brillouin zone, it therefore corresponds to the parabolic shape of the empty lattice approximation.





\section{Interaction of light with atoms}

\begin{marginfigure}
\inputtikz{\currfiledir tls_absorption}
\caption{A light beam induces a transition from $\ket{i}$ to the  $\ket{f}$.}
\end{marginfigure}
I would like to discuss how the coupling between two quantum mechanical systems affects their optical absorption spectrum. Before doing so, we need to set the terms for how quantum mechanics describes light-matter interaction.

Fermi's Golden Rule gives the transition rate from the initial state $\ket{i}$ to the final state $\ket{f}$ caused by any time-dependent perturbation $H'$ to the stationary Hamilton operator $H_0$ as
\begin{equation}
 \Gamma_{i \rightarrow f} = \frac{2 \pi}{\hbar} \, \left| \bra{f} H' \ket{i} \right|^2 \, \rho(E) \quad ,
\end{equation}
where  $\rho(E) = d n / d E = \rho(\omega) / \hbar$ is the density of final states. The idea is that the initial state  $\ket{i}$ is well known, but the outcome of the interaction $\ket{f}$ might have free parameters, for example the direction of the emitted electron or the mode of the absorbed photon. The density of states   $\rho(E)$ thus can describe either electronic or photonic states, or both.




In general, the interaction of a charged particle with an electromagnetic vector potential $\mathbf{A}$ is described by the perturbation
\begin{equation}
 H' = - \frac{i \hbar e}{m} \, \mathbf{A \cdot \nabla}  \quad .
\end{equation}
As the spatial extent of our wavefunctions is small compared to the wavelength of light, we employ the dipole approximation and assume $\exp( i \mathbf{k \cdot r}) \approx 1$ in the plane-wave description of the vector potential. In this way, the  perturbation operator $H'$ simplifies to\sidenote{see \textcite{bransden_joachain} for details}
\begin{equation}
 H' =  e \, \mathbf{E} (t)  \mathbf{\cdot \, r} =  e \,E_0 \,  \mathbf{\hat{x} \cdot \, r} \, \cos(\omega t) \quad ,
\end{equation}
where $\mathbf{\hat{x}} $ is a unit vector defining the polarization direction of the light field. We simplify further by using the rotating-wave approximation and keeping only co-rotating parts 
\begin{equation}
 \cos(\omega t)
 = \frac{1}{2} \left( e^{i \omega t}+  e^{-i \omega t} \right)
 \approx  \frac{1}{2}  e^{i \omega t} 
\end{equation}
so that 
\begin{equation}
H' =  \frac{ e \,E_0}{2}  \,  \mathbf{\hat{x} \cdot \, r} \,  e^{i \omega t}  \quad .
\end{equation}
%
 We introduce the transition dipole matrix element $\mu_{if}$ as
\begin{equation}
\mathbf{\mu}_{if} = -e \, \bra{f}    \mathbf{r} \ket{i}  \quad .
\end{equation}
It has the units of an electric dipole moment, i.e., charge times distance, and is the central element of an optical transition in quantum mechanics. For practical reasons, one uses the unit of 1~Debye = 1 electron displaced by 0.208~\AA.
With this the matrix element  becomes
\begin{equation}
\left| \bra{f} H' \ket{i} \right|^2 =  \frac{1}{4} E_0^2  \, |\mathbf{\hat{x}} \cdot \mathbf{\mu}_{if} |^2 \quad .
\end{equation}
Plugging everything into Fermi's Golden Rule, we get
\begin{equation}
 \Gamma_{i \rightarrow f} = \frac{\pi}{2 \hbar^2} \,  E_0^2  \, |\mathbf{\hat{x}} \cdot \mathbf{\mu}_{if} |^2 \, \rho(\omega) \quad .
\end{equation}
Now we have to take into account that we use a incoherent multimode light source. The electric field $E$ is here an incoherent superposition of  modes with the  spectral energy density $u(\omega)$.\sidenote{see \textcite{CT} and \textcite{Fox}   for details}
The total power is thus
\begin{equation}
 \frac{1}{2} \epsilon_0  \, E_0^2  = \int  u(\omega)  \, d\omega \quad .
\end{equation}
The  transition rate is thus 
\begin{equation}
 \Gamma_{i \rightarrow f} =   \frac{\pi  }{\hbar^2 \epsilon_0}  \, |\mathbf{\hat{x}} \cdot \mathbf{\mu}_{if} |^2 \,
\int u(\omega)  
  \rho(\omega)  d \omega \quad .
\end{equation}
As the atomic transition is narrow compared with the light spectrum, the density of states $\rho(\omega)$ selects the transition frequency $\omega_{if}$ 
\begin{equation}
 \Gamma_{i \rightarrow f} =   \frac{\pi  }{\hbar^2 \epsilon_0}  \, |\mathbf{\hat{x}} \cdot \mathbf{\mu}_{if} |^2 \,
 u(\omega_{if})   \quad .
\end{equation}
Both the absorption and the emission spectrum of an atom or a molecule is connected to the transition rate $\Gamma_{i \rightarrow f}$ via the Einstein coefficients, but with different pre-factors:
\begin{align}
   \text{absorption} & \propto \omega_{if} \  |\mathbf{\mu}_{if} |^2 \\
   \text{emission} & \propto \omega_{if}^3 \  |\mathbf{\mu}_{if} |^2  \quad .
\end{align}



\section{Transfer of excitation from one molecule to another}

In a coupled pendulum, energy is transferred from one pendulum to the other (and back again). Here we will investigate this transfer of excitation for a quantum mechanical system\sidenote{More in \cite{KoehlerBaessler2015} and  \cite{Valeur_mol_fl}}. We have two molecules and each molecule has a wave function of the ground state ($a$, $b$) and of an excited state ($a^\star$, $b^\star$). Two electrons (1,2) are involved, but we can not distinguish them, so that we need to construct the usual anti-symmetric wavefunctions. Initially, molecule $a$ should be excited and $b$ in the ground state. The initial wavefunction is thus
\begin{equation}
   \psi_i = \frac{1}{\sqrt{2}} \left( \psi_{a^\star}(1) \,  \psi_{b}(2) -   \psi_{a^\star}(2) \,  \psi_{b}(1)\right)  \quad .
\end{equation}
In the final state, the excitation should have swapped, i.e.
\begin{equation}
   \psi_f = \frac{1}{\sqrt{2}} \left( \psi_{a}(1) \,  \psi_{b^\star}(2) -   \psi_{a}(2) \,  \psi_{b^\star}(1)\right)  \quad .
\end{equation}
The two electrons will interact by their Coulomb potential. This is sufficient, as we will see, to swap the excitation. The transition matrix element or interaction energy $J$ is
\begin{equation}
   J = \frac{1}{4 \pi \epsilon_0} \left\langle \psi_f \middle| \frac{e^2}{r_{12}} \middle| \psi_i \right\rangle \quad ,
\end{equation}
where $r_{12}$ is the distance of the electrons. Multiplying this out, we get two pairs of terms. In each pair, electron 1 and 2 change role. One pair has the form
\begin{equation}
   J^C = \frac{2}{4 \pi \epsilon_0}  \left\langle  \psi_{a^\star}(1) \,  \psi_{b}(2)\middle| \frac{e^2}{r_{12}} \middle|  \psi_{a}(1) \,  \psi_{b^\star}(2)\right\rangle \quad ,
\end{equation} 
i.e., the electrons stay on 'their' molecule but change between excited and ground state. This is called the Coulomb term. In the second pair
\begin{equation}
   J^E = \frac{2}{4 \pi \epsilon_0}  \left\langle \psi_{a^\star}(1) \,  \psi_{b}(2) \middle| \frac{e^2}{r_{12}} \middle|  \psi_{a}(2) \,  \psi_{b^\star}(1) \right\rangle
\end{equation} 
%\left\langle \frac{1}{2} \middle| 1 \right\rangle
%
the electrons change molecule and take their state (ground vs excited) with them. This is the exchange term. It requires a bond between the molecules for the electrons to move. In the following, we will only look at the first term, the Coulomb term, that acts 'via the air' and drop the $C$ in $J^C$.





\section{Coupling of two transition dipole moments}

The transition matrix element $J^C$ is similar to two charge densities that interact. It was the work of Dexter and Förster to apply a multipole-multipole expansion to simplify things. We keep only the lowest term, the dipole-dipole contribution. The dipoles are transition dipole moments of the form
\begin{equation}
   \boldsymbol{\mu}_a = \braket{ \psi_{a^\star}(1)| e \, \br |  \psi_{a}(1) }
\end{equation}
i.e., an electron changes from ground to excited state.
%
% We consider two molecules, $a$ and $b$, each with a ground ($0$) and an excited ($1$) state. We write the wave function in the form $\ket{ab}$, i.e. $\ket{01}$ is molecule $a$ in the ground state, molecule $b$ in the excited state. In each molecule an optical transition dipole moment couples the ground and excited states, i.e. $\braket{10| \hat{\mu}_a | 00}$ and $\braket{01| \hat{\mu}_b | 00}$ are different from zero and describe an excitation of molecule $a$ and $b$ respectively. In addition, the two transition dipole moments interact and lead to a resonant coupling of the 
% $\ket{01}$ and $\ket{10}$ states\footcite{knoester-book}
% \begin{equation}
% \hat{H}_{coupling} = J \left( \ket{10}\bra{01} + \ket{01}\bra{10} \right) \quad .
% \end{equation}
The coupling energy $J$ depends on the distance $\boldsymbol{r}_{ab}$ and the relative orientation of the transition dipoles $\boldsymbol{\mu}_{a,b}$. It can be thought of as the energy of one dipole in the field of another.
\begin{eqnarray}
 J & = & \frac{1}{4 \pi \epsilon_0}  \left( \frac{\boldsymbol{\mu}_a \cdot \boldsymbol{\mu}_b }{|\boldsymbol{r}_{ab}|^3} 
  - 3 \frac{ (\boldsymbol{\mu}_a \cdot \boldsymbol{r}_{ab}) (\boldsymbol{\mu}_b \cdot \boldsymbol{r}_{ab})
  }{ |\boldsymbol{r}_{ab}|^5 }  \right)\\
   & = & \frac{\mu_a \mu_b }{4 \pi \epsilon_0 \, r_{ab}^3} \left( \cos \theta - 3 \cos \alpha \, \cos \beta \right) = \frac{\mu_a \mu_b }{4 \pi \epsilon_0 \,r_{ab}^3} \, \kappa  
\end{eqnarray}
where the angles are defined in the sketch.

\begin{marginfigure}
   \inputtikz{\currfiledir/angles}

\caption{Sketch showing 
The angles used to calculate the coupling factor $\kappa$.}
\end{marginfigure}

We now consider three states: both molecules in the ground state, molecule $a$ excited ($\psi_i$ of last section), and molecule $b$ excited ($\psi_f$). In matrix form, the Hamilton operator reads 
\begin{equation}
\hat{H} = \begin{pmatrix}
0 & \mu_a \mathcal{E}& \mu_b \mathcal{E} \\
\mu_a^\star \mathcal{E}^\star & \hbar \omega_a & J  \\
\mu_b^\star \mathcal{E}^\star & J^\star & \hbar \omega_b \\
\end{pmatrix} \quad .
\end{equation}
The transition dipole moments together with an external optical field $\mathcal{E}$ couple the ground state to the excited states. The excited states are coupled by dipole-dipole interaction without the need for an external field. For example, by measuring an absorption spectrum starting from the ground state, we can determine the energies of the excited states, which we can find by diagonalizing the lower $2 \times 2$ matrix as discussed above.

The coupling results in new states, which can be written as linear combinations of the uncoupled states. This has an effect on the observables. When $\ket{\psi}$ is a linear combination of $\psi_a$ and $\psi_b$, then also the transition dipole moment from  the ground state to $\ket{\psi}$ is a linear combination of $\mu_a$ and $\mu_b$ with the same weights. When $J \gg |E_a - E_b| / 2$ then we get (see Fig. \ref{fig:6_anticrossing})
\begin{equation}
 \boldsymbol{\mu}_{\pm} = \sqrt{1/2} \, \left( \boldsymbol{\mu}_a \pm \boldsymbol{\mu}_b \right) \quad .
\end{equation}
The brightness of the absorption line is for identical molecules, i.e. $\mu = \mu_a = \mu_b$
\begin{equation}
 I \propto |\boldsymbol{\mu}_{\pm}|^2 = (1/2) \, \left| \boldsymbol{\mu}_a \pm \boldsymbol{\mu}_b \right|^2 = \left( 1 \pm \cos \theta \right) \, \left| \boldsymbol{\mu}   \right| ^2 \quad ,
\end{equation}
where $\theta$ is as above the angle between the transition dipole moments.  

The spectroscopic signature of coherent coupling between two molecules is thus a splitting of the absorption line into two lines separated by twice the coupling energy $J$. The sum of the line amplitudes remains unchanged, but in some cases (H- and J-aggregates, see below) one transition takes up the whole amplitude and the other remains dark. In these cases there is no splitting but a shift of the absorption line. The coupling disappears when both dipoles are perpendicular to each other ($\theta = 90^\circ$).

\section{H- and J-aggregates}

\begin{marginfigure}
   \inputtikz{\currfiledir jh}

\caption{J- and H aggregates.}
\end{marginfigure}

Two important limiting cases are the H- and J-aggregates.\footcite[chapters 2.1.4.3, 2.2.5.3]{KoehlerBaessler2015} In a J-aggregate the dipoles are oriented parallel and head-to-tail, i.e. $\alpha = \beta = \theta = 0$ and therefore $\kappa = -2$. A negative $\kappa$ implies that the coupling constant $J$ is negative. The state $\Psi_+$, which carries all the oscillator strength, has an energy $E_+ = (E_a + E_b) / 2 + J$, which is lower than the average energy of the uncoupled states. The absorption line therefore shifts towards the red. The same applies to the fluorescence emission spectrum.

In an H-aggregate the dipoles are also parallel, but side by side, i.e. $\alpha = \beta = 90^\circ$ and $\theta = 0$. In this case $\kappa =1$ and $J$ is positive. The absorption line shifts to blue when aggregates form, as the $\Psi_+$ state again gets all the oscillator power. However, as fluorescence emission is slow compared to other relaxation processes, this high energy state does not emit light. H-aggregates appear dark in the emission.

\begin{marginfigure}
   \inputtikz{\currfiledir TDBC}

\caption{Absorption spectrum of TDBC dye in solution. When increasing the concentration (thick), more monomers aggregate. (Data by T. Kroh, 2014) }
\end{marginfigure}


The width of the absorption line of a dye at room temperature is determined by dephasing, i.e. fluctuations in the environment that are fast compared to the lifetime of the excited state, and by static differences in the environment of different chromophores. The spectral position of the absorption line in a molecular aggregate is the average of two single chromophore transitions. As in the propagation of uncertainties in an experiment, the width of the new distribution, generated as an average over two values from the old distribution, is reduced by a factor of $\sqrt{2}$. This applies more generally\footcite{Knapp1984}, so that an aggregate of $N$ chromophores is expected to have a spectral line width reduced\sidenote{This is the same physics as motion narrowing in NMR.} by $\sqrt{N}$.



%\begin{tabular}{llll}
%$\theta$ & $0^\circ$ & $90^\circ$ & $0^\circ$ \\
%$\alpha$ & $0^\circ$ & $0^\circ$ & $90^\circ$ \\
%$\beta$ & $0^\circ$ & $90^\circ$ & $90^\circ$ \\
%$\kappa$ & $-2 $ & $ 0 $ & $1$ \
%\end{tabular}




%--------------------
\printbibliography[segment=\therefsegment,heading=subbibliography]


