\renewcommand{\lastmod}{December 1, 2023}
\renewcommand{\chapterauthors}{Markus Lippitz}


\chapter{Lattice of plasmonic particles}

\section{Overview}

Now we combine everything. We have three ingredients: the plasmonic resonance of a small silver particle, a periodic arrangement of these particles in a two-dimensional lattice, and the narrow optical resonance of the TDBC dye. The particles hybridized with each other, combining the plasmon hybridization of Chapter \ref{chap:hybrid_classic} with lattice modes, as in a chain of masses and in solid-state physics in general. As we will see below, this results in an X-shaped dispersion relation, which we measure using Fourier optics.

In a second step, this lattice plasmon mode hybridizes with the optical transition of the dye, leading to characteristic anti-crossing features in the dispersion relation, where the X crosses the horizontal dye dispersion. Microscopically, we can model the influence of the dye by taking into account the refractive index of the medium.

This chapter is based on the master thesis of Simon Durst\footcite{Durst21}. All figures are taken from his thesis, sometimes slightly modified.


\begin{figure}
  \includegraphics[width=\textwidth]{\currfiledir sample_dye.pdf}
  \caption{Dispersion relation of a lattice plasmon hybridized (X-shaped) with a dye (horizontal), for different concentrations of the dye (left to right: 1, 10, 60 weight \%). The grey scale gives the extinction. \label{fig:8_intro} }
  \end{figure}
  

  

\section{How this is  measured}

The samples consist of a rectangular lattice of silver particles on a glass substrate. It is fabricated by electron beam lithography. An electron sensitive resist is exposed in a (slightly modified) electron microscope and then developed. At the exposed areas, the glass surface is accessible. The unexposed areas are still covered by the resist.  A gold or silver film is deposited by thermal evaporation. When the resist is chemically removed, it takes with it the metal film on the unexposed areas. The metal remains only where it adheres directly to the glass surface. In this way, arbitrary two-dimensional structures can be fabricated with a resolution limited by the electron beam to about 50--80 nm in our case.

We have chosen a particle size of $80 \times 80$ nm. The lattice constant $p_y$ is always 200 nm, $p_x$ varies (see below). The lattice is coated with a polymer film containing the TDBC dye in variable concentration. Everything is covered with immersion oil ($n=1.5$) and a second glass substrate, so that the particles are embedded in a homogeneous dielectric environment.


\begin{marginfigure}
\includegraphics[width=30mm]{\currfiledir sample.pdf}
\caption{SEM micrograph of gold nanorods in a lattice. Indicated are the dimensions of the rods and the lattice constants.}
\end{marginfigure}


For a dispersion relation we need an energy $E$ and a wave vector $k$. Since the sample is only two-dimensional, this is an in-plane wave vector, which we have chosen to be along the x-direction, i.e. $k_x$. The signal measured as the function of $E$ and $k_x$ is the transmission $T$ or the extinction\sidenote{In this sample, not only absorption but also scattering leads to a reduced transmission. Therefore, we call it extinction and not absorption.} $1-T$.
We measure transmission spectra for white light as a function of angle of incidence. Electron beam lithography results in a finite size lattice of $30 \times 30$ \textmu m. It is difficult to keep the sample in the beam when rotating either the sample or the beam. So we keep everything fixed and measure all the angles at the same time. To do this, we illuminate the sample with a light cone with a large aperture angle (NA = 0.9, max. angle = $64^\circ$). The transmitted light is collected by a second microscope objective. Fourier optics tells us that the front focal plane is Fourier transformed into the back focal plane. We can no longer use a small angle of incidence, but an ideal imaging system must satisfy the Abbé-Sine condition that rays of equal angle of incidence $\theta$ intersect at the same point at a height $h$ with
\begin{equation}
 h = f \sin \theta \quad .
\end{equation}

\begin{marginfigure}
\includegraphics[width=45mm]{\currfiledir bfp.pdf}
\caption{The back focal plane sorts rays by their angle in the front focal plane.}
\end{marginfigure}

We image this back-focal plane (BFP) on the entrance slit of a spectrometer and thus get an image on the CCD camera which in one direction is angle of incidence, in the other wavelength of the light beam. This is converted into a $E$-$k_x$ scale. As the maximum angle is fixed, but $k \propto E$, we observable region of the dispersion relation has a trapezoidal shape, i.e, the maximum value of $k_x$ is lower at lower energy. The light beam is polarized before the sample such that we image the s polarization on the entrance slit of the spectrometer, i.e. along the x-direction of the sample coordinate system.


\begin{figure}
\includegraphics[width=\textwidth]{\currfiledir SetupTransmission_v3.pdf}
\caption{Setup to measure angle-dependent transmission spectra without moving parts. }
\end{figure}


\section{How to understand the dispersion relation (without dye)}

\begin{figure}
  \includegraphics[width=\textwidth]{\currfiledir field_ag2.pdf}
  \caption{Angle-dependent extinction spectrum of an array of plasmonic particles. \label{fig:8_without_dye} }
  \end{figure}
  

  

Each panel in Fig. \ref{fig:8_without_dye} shows a dispersion relation, i.e., the relation between (in-plane) momentum and energy. Two features combine: the particle plasmon resonance and a lattice resonance. The particle resonance is at a given eigen-frequency (or energy), spectrally rather broad, and independent of the angle of incidence or $k_x$, as the particles are rather spherical. This gives the broad, medium gray band around an energy of 2~eV. The second feature is the lattice resonance.  An optical wave travels parallel to the interface. Its dispersion relation is
\begin{equation}
 E =   \hbar c_0  \, \left(  k_x + m \cdot G \right)
\end{equation}
where $G = 2 \pi / p_x$ is the fundamental reciprocal lattice vector and $m$ an integer. This results in the X-shaped feature for $m = \pm 1$. With varying $p_x$ the crossing point, i.e., the energy $E$ at $k_x = 0$ varies, as
\begin{equation}
  E(k_x = 0) = m \frac{h c_0}{p_x} \quad .
\end{equation}
In the right-most panel of  Fig. \ref{fig:8_without_dye} we see the second order $m= \pm 2 $.

The eigenmode of the lattice resonance is spatially extended over the whole lattice and spectrally narrow. The interaction of a broad (particle)  and a narrow (lattice) resonance leads to characteristic spectral features that are visible in Fig. \ref{fig:8_without_dye}  and could be described as Fano resonance\sidenote{Ugo Fano, 1912--2001 }. But that is beyond the scope of this chapter. Here we follow a more microscopic approach. We calculate the extinction spectrum of an  arrangement of  many small particles at positions $\mathbf{r}_i$.
Each  particle is modelled as sphere  with a  polarizability $\alpha_i$ given by the material properties and the volume of the small particle. 

\section{Radiating electric dipole}

Let us first look at a single electric dipole $\boldsymbol{\mu}$ at the position $\mathbf{r}_0$. Its field at the position $\mathbf{r}$ is given by\footcite[eq. 8.52]{Novotny-Hecht2012} 
\begin{equation}
\mathbf{E}(\mathbf{r}) = \frac{k^2}{\epsilon_0 \, \epsilon_{out}} \, \mathbf{G}(\mathbf{r}, \mathbf{r}_0) \,  \boldsymbol{\mu}
\end{equation}
with the length $k$ of the wave vector in the medium of dielectric function $\epsilon_{out}$.
The Greens function $\mathbf{G}$ is given by\sidenote{This follows \cite{Novotny-Hecht2012}  eq. 8.55 and differs by $4 \pi k^2$ from eq. 2 in \cite{Yurkin07}}
\begin{equation}
\mathbf{G}(\mathbf{r}, \mathbf{r}_0) = \frac{e^{i k R} }{4 \pi \, k^2 \, R^3 } 
\left[  
 \left( k^2 R^2 + i k R  - 1 \right) \mathbf{1}  +    
  \left( 3 - 3 i k R - k^2 R^2  \right) \frac{\mathbf{RR}}{R^2}   
  \right] \label{eq:8_greens_dipole}
\end{equation}
with $\mathbf{R} = \mathbf{r} - \mathbf{r}_0$, $R = |\mathbf{R}|$, $\mathbf{1}$ the unity $3 \times 3$-tensor, and $\mathbf{RR}$ the outer product of $\mathbf{R}$ with itself, i.e.
\begin{equation}
\mathbf{RR} = 
\begin{pmatrix}
R_x R_x &  R_x R_y & R_x R_z \\
R_y R_x &  R_y R_y & R_y R_z \\
R_z R_x &  R_z R_y & R_z R_z \\
\end{pmatrix} \quad .
\end{equation}
This is a convenient method to describe the full vectorial field emitted by a dipole at position  $\mathbf{r}_0$ everywhere in space, including both near- and far-field components.

\begin{questions}
\item Convince yourself that eq. \ref{eq:8_greens_dipole} agrees with the other commonly used form 
\begin{equation}
  \mathbf{E} = \frac{ e^{i \, k  r} }{4\pi\epsilon_0 \, \epsilon_{out}}  \frac{1}{r^3}\left\{
      (k r )^2 \left( \hat{\mathbf{r}} \times \boldsymbol{\mu} \right) \times \hat{\mathbf{r}} +
      \left( 1 -  i k r \right)
        \left( 3\hat{\mathbf{r}} \left[\hat{\mathbf{r}} \cdot \boldsymbol{\mu}\right] - \boldsymbol{\mu} \right)
    \right\} \quad .
\end{equation}
\end{questions}

\section{Scattering sphere}

We need to know the oscillation amplitude of the emitting dipole $\mathbf{p}$ to use eq. \ref{eq:8_greens_dipole}. As in chapter \ref{chap:hybrid_quantum}, it is related by the  polarizability $\alpha$  to the incoming optical field $\mathbf{E}_{inc}$
\begin{equation}
\mathbf{p} = \epsilon_0 \, \epsilon_{out} \, \alpha \, \mathbf{E}_{inc}
\end{equation}
with the dielectric function $\epsilon_{out}$ of the embedding medium. 
When we assume that the scattering nanoobject is a sphere, we can calculate
\begin{equation}
 \alpha = 3V \, \frac{\epsilon_{in} - \epsilon_{out}}{\epsilon_{in} + 2 \epsilon_{out}} \quad ,
\end{equation}
where $V$ is the volume of the sphere and  $\epsilon_{in}$ the dielectric function of it. 
The sphere  radiates a scattered field $\mathbf{E}_S$
\begin{align}
\mathbf{E}_S(\mathbf{r}) & =  \frac{k^2}{\epsilon_0 \, \epsilon_{out}} \, \mathbf{G}(\mathbf{r}, \mathbf{r}_0) \,  \mathbf{p} \\
 & =  \frac{1}{4 \pi \, \epsilon_0 \, \epsilon_{out}  }  \frac{e^{i k R} }{  R^3 } 
\left[  \dots \right] \, \mathbf{p}
\end{align}
where the contents of the square brackets is the same as in eq.\ref{eq:8_greens_dipole} above. $k$ is again the length of the wave vector in a medium with dielectric function $\epsilon_{out}$.

\section{Multiple particles}

When we have more than one particle, each particle $i$ sees the incident field $\mathbf{E}_{i, inc}$ at the   position $\mathbf{r}_i$ plus the sum over all scattered fields $\mathbf{E}_{j, S}$ from all the other induced dipoles $j$
\begin{equation}
\mathbf{E}_{i, loc} = \mathbf{E}_{i, inc} + \sum_{j \neq i} \mathbf{E}_{j, S}
 = \mathbf{E}_{0} \, e^{i \mathbf{k} \cdot \mathbf{r}_i} \, +  \, 
 \sum_{j \neq i} \frac{k^2}{\epsilon_0 \, \epsilon_{out}} 
 \mathbf{G}(\mathbf{r}_i, \mathbf{r}_j) \,  \mathbf{p}_j  \label{eq:8_elocal}
\end{equation}
with the dipole moment $ \mathbf{p}_j$ of the particle at position $\mathbf{r}_j$. The position of the 'receiving' particle $\mathbf{r}_i$ takes the role of $\mathbf{r}$ in the Greens function; the position of the scattering particle $\mathbf{r}_j$ takes the role of the dipole at position $\mathbf{r}_0$ above.

The local field $\mathbf{E}_{i, loc}$ then induces a dipole moment again 
\begin{equation}
\mathbf{p}_i = \epsilon_0 \, \epsilon_{out} \, \alpha_i \, \mathbf{E}_{i,loc} \quad .
\end{equation}
Both equations together form a coupled equation system for the 
 dipole moments $ \mathbf{p}_i$
\begin{equation}
\mathbf{E}_{0} \, e^{i \mathbf{k} \cdot \mathbf{r}_i} =
\frac{1}{\epsilon_0 \epsilon_{out} \alpha_i} \, \mathbf{p}_i 
 \,  - \, 
 \sum_{j \neq i} \frac{k^2}{\epsilon_0 \, \epsilon_{out}} 
 \mathbf{G}(\mathbf{r}_i, \mathbf{r}_j)  \, \mathbf{p}_j 
\end{equation} 
 which can be written as
 \begin{equation}
 \mathbf{E}_{inc} = \mathbf{A} \, \mathbf{p} \quad , \label{eq:8_eq_system_EAP}
 \end{equation}
where $\mathbf{p}$ and $ \mathbf{E}_{inc} $ are column vectors containing the induced dipole moment and the incident field of all dipoles and $\mathbf{A} $ is an interaction matrix. Its elements are $3 \times 3$-sub-matrices given by\sidenote{This assume an isotropic polarizability. Otherwise, the diagonal elements should be $1/\alpha_{x,y,z}$ instead of $\mathbf{1}  / \alpha$.}
 \begin{align}
 \mathbf{A}_{ii} = &\frac{1}{\epsilon_0 \epsilon_{out} \alpha_i} \, \mathbf{1} \\
 \mathbf{A}_{ij} = & - \, 
 \frac{k^2}{\epsilon_0 \, \epsilon_{out}} 
 \mathbf{G}(\mathbf{r}_i, \mathbf{r}_j)  \quad .
 \end{align}
Some publications put the minus sign of the last equation into the Greens function. 


The extinction cross-section can be calculated by the optical theorem from the interference of the forward-scattered wave with the incident wave. We get\footcite{Draine88,Yurkin07}
\begin{equation}
\sigma_{ext} = \frac{k}{\epsilon_0 \epsilon_{out}  \, |\mathbf{E}_{inc}|^2} \, \sum_i \, \Im ( \mathbf{p}_i \cdot \mathbf{E}_{i, inc}^\star ) \quad .
\end{equation}
The term $\Im ( \mathbf{p} \cdot \mathbf{E}_{inc}^\star )$ is very similar to the term for absorption, in which not the incident field $\mathbf{E}_{inc} $ but the local field $\mathbf{E}_{loc} $ would be used.

 

 
With this we have now everything at hand to calculate the  extinction spectra of arrangements  of small scattering spheres or dipoles. We solve eq. \ref{eq:8_eq_system_EAP} for $\mathbf{p}$ and then calculate the cross-section. Depending on the community (and the distance between the dipoles) this is called discrete dipole approximation (DDA) or coupled dipole approximate (CDA).
 
 
\begin{questions} 
  \item Which size / dimension have $ \mathbf{A}$ and $\mathbf{p}$ in eq. \ref{eq:8_eq_system_EAP} ?
  \item Sketch the interaction matrix $ \mathbf{A}$ and its components.
  \end{questions}
  
 

\section{Lattice sum}

Things become easier when we are interested in infinite lattices of identical scatterers. As we are on a lattice, all lattice points are equal, especially in the amplitude and vectorial direction $\mathbf{\hat{n}}$ of the local field. It is then 
 convenient to re-arrange eq. \ref{eq:8_elocal}
\begin{equation}
\mathbf{E}_{i, loc} =\mathbf{E}_{0} \, e^{i \mathbf{k} \cdot \mathbf{r}_i} \, +  \, 
 \sum_{j \neq i} k^2 \, 
\mathbf{G}(\mathbf{r}_i, \mathbf{r}_j) \,    \alpha \, \mathbf{E}_{j,loc}
\end{equation}
to
\begin{equation}
E_{i, loc}  \, e^{-i \mathbf{k} \cdot \mathbf{r}_i} =\mathbf{\hat{n}} \cdot \mathbf{E}_{0}  +  \, 
 \sum_{j \neq i} k^2 \, 
 \mathbf{\hat{n}} \mathbf{G}(\mathbf{r}_i, \mathbf{r}_j)  \mathbf{\hat{n}}\,    \alpha \, E_{j,loc} \, e^{-i \mathbf{k} \cdot \mathbf{r}_i} \,
\end{equation}
so that we get
\begin{equation}
\mathbf{\hat{n}} \cdot \mathbf{E}_{0} = 
E_{loc} \left( 1 -     \alpha  \,
 \sum_{j \neq i} k^2 \, 
 \mathbf{\hat{n}} \mathbf{G}(\mathbf{r}_i, \mathbf{r}_j)  \mathbf{\hat{n}}\,     \, e^{i \mathbf{k} \cdot ( \mathbf{r}_i - \mathbf{r}_j  ) } \right)
 = 
 {E}_{loc} \left( 1 -     \alpha  \, S \right)
\end{equation}
with the retarded lattice sum $S$. The induced
dipole moment becomes 
\begin{equation}
\mathbf{p} = \epsilon_0 \, \epsilon_{out} \, \alpha \, \mathbf{E}_{loc} =  \epsilon_0 \, \epsilon_{out} \, \frac{\alpha}{ 1 -     \alpha  \, S } \,
  \mathbf{E}_{0} 
\end{equation} 
or  we define an effective (lattice) polarizability
\begin{equation}
\alpha_\text{lattice} = \frac{\alpha}{ 1 -     \alpha  \, S }  \quad .
\end{equation} 
The extinction cross-section then becomes\sidenote{somehow a $4 \pi$ is missing here....}
\begin{equation}
\sigma_{ext} = k \, \Im(\alpha_\text{lattice})  \quad . \label{eq:8_lattice_ext}
\end{equation}
The simulated dispersion relations agree well with the measured ones, as Fig.\ref{fig:8_sim_without_dye} shows.

\begin{figure}
  \includegraphics[width=\textwidth]{\currfiledir sim_field_ag2.pdf}
  \caption{Simulations compared to experiment. \label{fig:8_sim_without_dye} }
\end{figure}
  

  \section{Coupling to a dye}
  Now that we have described the dispersion relation of the plasmonic lattice, we also include the TDBC dye. The dye molecules are embedded in the polymer film on top of the sample. The film covers the plasmonic particles and the area between them. The thickness of the film was chosen to be about 50~nm, roughly the height of the plasmonic nanostructure. The plasmonic near-fields are concentrated in a region of a few 10 nanometers around the particle. A film that is too thick would contain many dye molecules that are too far away to interact with the lattice plasmon.

  The dye film leads to strong absorption around an energy $E \approx 2.1$~eV that is independent of the in-plane wave vector $k_x$. To better visualize the effect of the plasmon-dye coupling, we subtract the extinction of a dye film without plasmonic particles. This is shown in the figure \ref{fig:8_intro} at the beginning of this chapter. We see a reduced absorption at the dye resonance energy, i.e. a dark horizontal line in the dispersion plot. In the 'particle plus dye' sample, we have less dye because some volume is taken up by the particle. This effect reverses the sign of the dye feature at 2.1~eV.
  
  Apart from the sign of the dye feature, we observe that the lattice plasmon feature changes from an X shape. When the lattice plasmon crosses the dye resonance, the lattice resonance bends away. This bending can be observed especially at the highest dye concentration (right panel). At higher energies, the lattice plasmon feature continues unperturbed.

  We can model the coupling between the lattice plasmon and the dye using the usual coupled oscillator approach. The uncoupled oscillators have the eigenenergies
\begin{align}
  E_\text{dye} = &  \text{const.} \\
  E_\text{LP}(k_x) = &  \hbar  \, \left(  k_x + m \cdot G \right) \quad .
\end{align}
With the coupling energy $J$ we have to diagonalize the matrix
\begin{equation}
  \begin{pmatrix}
    E_\text{dye} & J \\
    J & E_\text{LP}(k_x) 
  \end{pmatrix}
\end{equation}
and get
\begin{equation}
  E_\pm = \frac{E_\text{dye}  +E_\text{LP}(k_x)}{2}  \pm \sqrt{ \left( \frac{E_\text{dye} - E_\text{LP}(k_x)}{2} \right)^2 + J^2 } \quad .
\end{equation}
This model for $E_-$ is superimposed on the measured data in Fig.\ref{fig:8_sample_dye_anticross}. The blue symbols mark the minima of the extinction spectra at a given value of $k_x$ to indicate the dispersion of the coupled mode.\sidenote{In a Fano resonance, the position of the resonance is close to, but not always at, the minimum, which is ignored here.} Both agree well. We find coupling energies of 10, 40, and 86~meV, respectively.

  \begin{figure}
    \includegraphics[width=\textwidth]{\currfiledir sample_dye_anticross.pdf}
    \caption{Anticrossing of lattice plasmon and dye. The data is a zoomed version of Fig.\ref{fig:8_intro}. Superimposed are the lattice dispersion (green), the minima of the spectra (blue) and the coupled oscillator model (red). \label{fig:8_sample_dye_anticross} }
  \end{figure}
    

\section{Dielectric model}

The model discussed above is fine, but we could look at this experiment from a completely different direction. We ignore all quantum mechanics, two-level systems, and coupling, and use only Maxwell's equations. We consider the TDBC dye and its polymer matrix as a dielectric material surrounding the plasmonic lattice. We get its index of refraction from the absorption spectrum, related to the imaginary part of the refractive index of the film. 

The extinction coefficient $\alpha$ describes how the intensity of a beam decreases as it travels through absorbing media.
\begin{equation}
  I(x) = I_0 T = I_0 e^{- \alpha x}
\end{equation}
with the transmission $T$.
It is wavelength depended and connected to the imaginary part of the index of refraction 
\begin{equation}
  \alpha(\lambda) = \frac{4 \pi n_2(\lambda)}{ \lambda}
\end{equation}
All together we get with a film thickness $d$
\begin{equation}
  n_2(\lambda) = - \frac{\lambda}{4 \pi \, d} \log T
\end{equation}
So we can measure $n_2(\lambda)$ and calculate the real part $n_1$ by a Kramers-Kronig relation. Then we set the dielectric function of the embedding medium $\epsilon_{out} = (n_1 + i n_2)^2$ and again use the formalism that led to eq. \ref{eq:8_lattice_ext}. Of course, $\epsilon_{out}$ now depends on the dye concentration.  The result of this simulation is shown in figure 
\ref{fig:8_sample_dye_sim}. It is in good agreement with the data. The depth of the horizontal pure dye absorption line is not fully recovered, which means that the ratio of interacting to non-interacting dye molecules in the model is wrong, or the reference film in the experiment differs a bit from the film at the plasmonic array.

The upper branch of the coupled mode has more contrast, is better visible in the simulation than in the experiment. In is not that it does not bend in the experiment, is just seems to fade out.\sidenote{See \cite{Wang2014} for an explanation.} This is not fully captured by the model, but the model also in general shows sharper features than the experiment.

  \begin{figure}
    \includegraphics[width=\textwidth]{\currfiledir sample_dye_sim.pdf}
    \caption{The left half of each panel shows the same data as Fig. \ref{fig:8_intro}. The right half is calculated using the lattice sum and a wavelength dependent dielectric function describing the dye.\label{fig:8_sample_dye_sim} }
  \end{figure}
    
We do not need to use a coupling model to describe the bending of the lattice plasmon dispersion. Classical electrodynamics is good enough. The microscopic description of matter in terms of quantum mechanics and oscillating transition dipoles has to smoothly merge into the description of classical electrodynamics, as required by the correspondence principle.




%-------------------


\printbibliography[segment=\therefsegment,heading=subbibliography]




